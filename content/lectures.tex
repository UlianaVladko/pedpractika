\section*{Модуль лекций: Многоагентные системы для магистров}

\subsection*{Описание модуля}
Модуль из 4 лекций по 1,5 часа предназначен для студентов магистратуры 1-го курса, изучающих аналитическое моделирование в междисциплинарных областях. Курс фокусируется на многоагентных системах (МАС) как инструменте моделирования сложных систем, интегрируя их с ранее изученными динамическими системами, системами управления и теорией очередей. Модуль включает теоретические основы, методы проектирования МАС и их применение в смешанных моделях (например, транспорт, энергетика, экономика).

\subsection*{Цели модуля}
\begin{itemize}
    \item Освоить принципы проектирования и анализа многоагентных систем.
    \item Научиться интегрировать МАС с динамическими системами, системами управления и теорией очередей.
    \item Разработать навыки моделирования МАС в междисциплинарных задачах.
    \item Подготовить студентов к реализации МАС в системах моделирования (AnyLogic, NetLogo, Python).
\end{itemize}

\subsection*{Целевая аудитория}
Магистры 1-го курса с базовыми знаниями в аналитическом моделировании, динамических системах, системах управления, теории очередей и программировании (Python или Java). Предполагается владение основами ИИ и системного моделирования.

% Lecture 1
\subsection*{Лекция 1: Основы многоагентных систем и их связь с аналитическими моделями}
\textbf{Цель}: Познакомить студентов с МАС и показать их место в аналитическом моделировании.

\textbf{Содержание}:
\begin{itemize}
    \item Определение МАС: агенты, их свойства (автономность, реактивность, проактивность, социальность).
    \item Сравнение МАС с другими моделями:
    \begin{itemize}
        \item Динамические системы: инерционность vs. автономность агентов.
        \item Системы управления: централизованное vs. децентрализованное управление.
        \item Теория очередей: МАС как инструмент моделирования потоков и взаимодействий.
    \end{itemize}
    \item Типы МАС: кооперативные, конкурентные, смешанные.
    \item Осовременить! Примеры междисциплинарных приложений: транспортные системы (управление трафиком), энергетика (умные сети), экономика (аукционы).
    \item Заметка! Роевой интеллект
    %\item Инструменты моделирования МАС: AnyLogic, NetLogo, SPADE.
\end{itemize}

% Lecture 2
\subsection*{Лекция 2: Архитектуры агентов и интеграция с системами управления}
\textbf{Цель}: Изучить архитектуры агентов и их применение в системах управления.

\textbf{Содержание}:
\begin{itemize}
    \item Архитектуры агентов:
    \begin{itemize}
        \item Реактивные: связь с динамическими системами.
        \item Делиберативные (BDI): моделирование сложного поведения.
        \item Гибридные: комбинация реактивности и планирования.
    \end{itemize}
    \item Интеграция с системами управления:
    \begin{itemize}
        \item МАС как децентрализованная САУ (система автоматического управления).
        \item Примеры: управление роботами, адаптивное управление в энергосетях.
    \end{itemize}
    \item Моделирование агентов в AnyLogic: обзор компонентов и библиотек.
    \item Проблемы: масштабируемость, синхронизация, обработка неопределенности.
\end{itemize}

% Lecture 3
\subsection*{Лекция 3: Взаимодействие и координация в МАС}
\textbf{Цель}: Рассмотреть механизмы взаимодействия агентов и их связь с теорией очередей.

\textbf{Содержание}:
\begin{itemize}
    \item Типы взаимодействий: кооперация, конкуренция, координация.
    \item Протоколы взаимодействия: FIPA-ACL, контрактные сети, аукционы.
    \item МАС и теория очередей:
    \begin{itemize}
        \item Моделирование потоков клиентов/ресурсов как взаимодействий агентов.
        \item Примеры: системы массового обслуживания в аэропортах, логистике.
    \end{itemize}
    \item Алгоритмы координации: распределенное планирование, консенсус.
    \item Практическая реализация: моделирование взаимодействия в NetLogo или Python (SPADE).
\end{itemize}

% Lecture 4
\subsection*{Лекция 4: МАС в междисциплинарных задачах и современные подходы}
\textbf{Цель}: Изучить применение МАС в смешанных моделях и новые направления.

\textbf{Содержание}:
\begin{itemize}
    \item МАС в междисциплинарных задачах:
    \begin{itemize}
        \item Транспорт: моделирование трафика с учетом динамических систем.
        \item Энергетика: умные сети с интеграцией теории очередей.
        \item Экономика: аукционы и конкурентные модели (теория игр).
    \end{itemize}
    \item Современные подходы:
    \begin{itemize}
        \item Многоагентное обучение с подкреплением (MARL) для адаптивных систем.
        \item Роевой интеллект: моделирование на основе биологических систем.
        \item Интеграция с IoT и блокчейн для децентрализованных систем.
    \end{itemize}
    \item Проблемы и вызовы: масштабируемость, этика, безопасность.
    \item Кейс: моделирование умного города с использованием МАС.
\end{itemize}