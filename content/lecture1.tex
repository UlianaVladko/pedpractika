\subsection*{Лекция 1. Введение в проблематику, основные понятия и определения.}

В области ИТ-технологий, и в частности,во множестве направлений искусственного интеллекта имеют место исследования, получившие название
“Многоагентные системы”. Исследования по интеллектуальным агентам и
многоагентным системам объединены в самостоятельный раздел искусственного интеллекта. Прикладной интерес к многоагентным системам обусловлен
достижениями в области информационных технологий, искусственного интеллекта, распределенных информационных систем, компьютерных сетей и в
компьютерной технике.

Многоагентные системы имеют реальную возможность интегрировать в
себе самые передовые достижения перечисленных областей, демонстрируя
принципиально новые качества. 

Первоначально идея интеллектуального посредника ("агента") возникла в связи с желанием упростить стиль общения конечного пользователя с компьютерными программами, поскольку в настоящее
время имеет место стиль взаимодействия пользователя с компьютером, основанный на том, что пользователь запускает задачу явным образом и управляет
ее решением.

Идея интеллектуального посредника возникла как попытка интеллектуализации пользовательского интерфейса, и развитие методов искусственного интеллекта позволило сделать новый шаг к изменению стиля взаимодействия пользователя с компьютером. Возникла идея создания так называемых
"автономных агентов", которые породили уже новый стиль взаимодействия
пользователя с программой. Вместо взаимодействия, инициируемого пользователем путем команд и прямых манипуляций, пользователь вовлекается в совместный процесс решения. 

При этом, как пользователь, так и компьютерный посредник, оба принимают участие в запуске задачи, управлении событиями и решении задачи.
Для такого стиля используется метафора "персональный ассистент" (ПА), который сотрудничает с пользователем в той же рабочей среде. 

Главная особенность интерфейса, обеспечиваемого ПА, состоит в том,
что этот интерфейс оказывается персонифицированным. Последнее достигается за счет того, что ПА наделяется способностью к обучению. В самом простом варианте, ПА получает информацию о привычках пользователя путем,
как говорят, "наблюдая" за работой своего пользователя. Обучаясь интересам,
привычкам и предпочтениям пользователя, а также окружающего его сообщества пользователей (это те, кто доступен персональному ассистенту через
компьютерную сеть), ПА может стать весьма полезным, причем в различных
аспектах: выполнять решение задач по поручению пользователя, тренировать
его, управлять событиями и процедурами. Заметим, что по существу персонификация пользовательского интерфейса - это новый резерв его интеллектуализации, который удачно дополняет “интеллектуальность интерфейса”, которая
традиционно ассоциируется только с экранными графическими средствами. 

Исследования и экспериментальные программные разработки довольно
быстро показали, что множество задач, в которых ПА с большой пользой может ассистировать пользователю, практически неограниченно: отбор информации, просмотр информации, поиск в Internet, управление электронной почтой, календарное планирование встреч, выбор книг, кино, музыки и т.д. Разработки в этой области поддерживались и поддерживаются такими известными фирмами, как Apple, Hewlett Packard, Digital, японскими фирмами. Термин
"персональный ассистент" была заменен на "интеллектуальный посредник",
или, как стали чаще говорить на русском языке - "интеллектуальный агент"
(ИА). 

В процессе эволюции данного понятия, приведенная выше идея вышла
за рамки интеллектуального пользовательского интерфейса, она все более и
более ориентировалась на идеи и методы искусственного интеллекта, на активное использование тех преимуществ, которые дают современные локальные и глобальные компьютерные сети, распределенные базы данных и распределенные вычисления. 

Активное развитие методов и технологий распределенного искусственного интеллекта, достижения в области аппаратных и программных средств
поддержки концепции распределенности и открытости привели к осознанию
того важного факта, что агенты могут интегрироваться в системы, совместно
решающие сложные задачи.

Это означало появление нового направления развития распределенных
систем искусственного интеллекта и такие системы получили название многоагентных систем.

“Агент — это инкапсулированная вычислительная система,
помещенная в некоторую среду и способная автономно выполнять
действия в этой среде для достижения поставленных целей.”
— Wooldridge and Jennings [WJ95]

“Агент — это вычислительная система, автономно действующая
от лица других сущностей, выполняющая действия реактивно
и/или с определенной целью и в некоторой степени
использующая свойства обучаемости, кооперативности и
мобильности.”
— Shaw Green et al. [GHN+97]

В самом общем случае агент обладает следующими свойствами:
\begin{itemize}
  \item реактивность: способность агента воспринимать окружающее и
влиять на него;
  \item целеустремленность: агент должен действовать в заложенными в
него целями;
  \item социальная активность: агент должен взаимодействовать с
другими агентами и/или людьми;
  \item автономность: агент действует без непосредственного
вмешательства человека и обладает определенным контролем на
своими действиями и внутренним состоянием.
\end{itemize}

Распространено использование ментальных характеристик:
\begin{itemize}
  \item знания,
  \item убеждения,
  \item намерения,
  \item обязательства и т.п.
\end{itemize}

Иногда агенты наделяются эмоциями.

Часто также используются следующие свойства агентов:
\begin{itemize}
  \item мобильность: способность агентов перемещаться
(физически или в сети);
  \item правдивость: предположение, что агент не может намеренно
фальсифицировать передаваемую (другим агентам, человеку)
информацию;
  \item доброжелательность: предположение, что цели агентов не
конфликтуют и, следовательно, каждый агент стремится
выполнить то, о чём его просят;
  \item рациональность: предположение, что агент действует в
соответствии со своими целями и не пытается противостоять себе
(по крайней мере, насколько это позволяют его убеждения).
\end{itemize}

Существует ряд теорий и концепций, использующихся для описания
агентов:
\begin{itemize}
  \item логические системы: цели и свойства агента описываются при
помощи высказываний в различных логических системах;
  \item системы намерений: внутреннее состояние агента представляется
системой мировоззрений (знания, убеждения, желания,
намерения и т.п.);
  \item модели коммуникаций: взаимодействие между агентами
происходит посредством специальных действий.
\end{itemize}

Определение: Многоагентная система - это множество интеллектуальных агентов, распределенных по сети, мигрирующих по ней в поисках релевантных данных, знаний и процедур и кооперирующихся в процессе выработки решений. 

Многоагентная система (МАС) — это система из нескольких
взаимодействующих интеллектуальных агентов.
МАС используются для:
\begin{itemize}
  \item регулирования траффика;
  \item онлайн-торговли;
  \item моделирования соц. структур;
  \item группового ИИ в играх и фильмах;
  \item устранения чрезвычайных ситуаций (ЧС);
сенсорных сетей и т.д.
\end{itemize}

В результате возникла новая концепция сообщества "программных роботов", цель которых - удовлетворение различных информационных и вычислительных потребностей конечных пользователей.

Структура исследований в области многоагентных систем в настоящее
время очень широка и сравнима с широтой исследований в области искусственного интеллекта.

Исследования в области многоагентных систем можно разделить на основные направления:

\begin{itemize}
  \item теория агентов, в которой рассматриваются формализмы и математические методы для описания рассуждений об агентах и для выражения желаемых свойств агентов;
  \item методы кооперации агентов (организации кооперативного поведения) в
процессе совместного решения задач или при каких-либо других вариантах
взаимодействия;
  \item архитектура агентов и многоагентных систем (область исследований,
в которой изучается, как построить компьютерную систему, которая удовлетворяет тем или иным свойствам, которые выражены средствами теории агентов);
  \item языки программирования агентов;
  \item методы, языки и средства коммуникации агентов;
  \item методы и программные средства поддержки мобильностиагентов
(миграции агентов по сети). 
\end{itemize}

Особое место среди этих направлений занимают исследования, связанные с разработкой приложений многоагентных систем и инструментальных
средств поддержки технологии их разработки. Можно еще упомянуть проблемы, связанные с аутентификацией (авторизацией) агентов, обеспечением
безопасности и ряд других.

В рамках изучения нашей дисциплины нами будут рассмотрены теория
агентов -> методы кооперации агентов -> архитектура агентов и многоагентных систем -> языки программирования агентов.

Что касается приложений многоагентных систем, то, в предлагаемомучебном пособии будет приведена отдельная лекция по приложениям данного
научного направления. 
